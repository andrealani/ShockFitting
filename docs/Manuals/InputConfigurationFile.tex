\documentclass[11pt,a4paper,oneside]{article}

\usepackage{graphicx}
\usepackage{subfigure}
\usepackage{longtable}
\usepackage{amsmath}
\usepackage{fancyhdr}
\usepackage{multirow}
\usepackage{color}
\usepackage{geometry}
\usepackage{rotating}
\usepackage{lscape}
\usepackage{setspace}
\usepackage{mathrsfs}
\usepackage{chemarr}
\usepackage{yfonts}
\usepackage{xfrac}

\begin{document}

\title{\underline{Shock-Fitting Solver: \textit{case} configuration file}}
\date{}
\maketitle
The Shock-Fitting Solver configuration file has the extension \textit{case}. 
It is composed by several lines. 
Each line is in the form \texttt{KEY = VALUE}.
The \texttt{KEY} is an object or an object parameter and the \texttt{VALUE} is the quantity assigned to \texttt{KEY}.
\newline
\newline
The \texttt{VALUE} can be:

\begin{itemize}
\item{an alpha-numeric string}
\item{an integer}
\item{a boolean (\textit{true} or \textit{false})}
\item{a floating point number}
\item{an arbitrary complex analytical function}
\item{an array of all the previous}
\end{itemize}

If a parameter is not necessary for a specified case, the \texttt{VALUE} can be set equal to "\texttt{/}".
The \texttt{VALUE}s can be broken in different lines by using the character backslash.
Comments start with "\#".


\subsubsection*{The \textit{Shock-fitting Solver} model}

\hspace*{1cm} \texttt{.ShockFittingObj = StandardShockFitting}
\newline
\newline
specifies the model of the Shock-fitting Solver. It corresponds to a set of functionalities defined inside the code.
Up-to-date the \texttt{StandardShockFitting} model and the next sections are, therefore, related to it.

\subsubsection*{Model setting}

\hspace*{1cm} \texttt{.StandardShockFitting.Version = original}
\newline
\newline
allows to choose the between different versions (if available) of the chosen Shock-Fitting Solver model. 
\newline
Up-to-date the \texttt{StandardShockFitting} has two version:

\begin{itemize}
\item{\texttt{original}: the \texttt{Triangle Mesh Generator} library is called as executable files. The data are passed to it through \texttt{I/O} files.}
\item{\texttt{optimized}: the \texttt{Triangle Mesh Generator} library is called through it functions. The data are passed to it through arrays.}
\end{itemize}
\vspace*{0.6cm}
\hspace*{1cm} \texttt{.StandardShockFitting.ResultsDir = $./$Results$\_$SF}
\newline
\newline
specifies the directory path to store the results.
\newline
\newline
\hspace*{1cm} \texttt{.StandardShockFitting.ComputeResidual = true}
\newline
\newline
specifies if the shock-fitting residual are computed during the execution. If \texttt{true}, the \texttt{ComputeResidual} object must be added to the \texttt{StateUpdaterSF} library list of section \ref{subsubsec:state updater}.
\newline
\newline
\hspace*{1cm} \texttt{.StandardShockFitting.startFromCapturedFiles = true}
\newline
\newline
defines if the Shock-fitting Solver initial files are generated from the files storing the captured solution.

\subsubsection*{MeshData}

\hspace*{1cm} \texttt{.StandardShockFitting.MeshData.EPS = 0.20e-12}
\newline
\hspace*{1cm} \texttt{.StandardShockFitting.MeshData.SNDMIN = 0.05}
\newline
\hspace*{1cm} \texttt{.StandardShockFitting.MeshData.DXCELL = 0.0006}
\newline
\hspace*{1cm} \texttt{.StandardShockFitting.MeshData.SHRELAX = 0.9}
\newline
\newline
define the distance between two shock faces, the maximum non-dimensional distance of phantom nodes, the length of the shock edges, the relax coefficient of shock points integration.
\newline
\newline
\hspace*{1cm} \texttt{.StandardShockFitting.MeshData.Naddholes = 0}
\newline
\newline
defines the number of hole points.
\newline
\newline
\hspace*{1cm}  \texttt{.StandardShockFitting.MeshData.CADDholes = 0}
\newline
\newline
defines the coordinates of the hole points specified above.
\newline
\newline
\hspace*{1cm}  \texttt{.StandardShockFitting.MeshData.freezedWallCells = true}
\newline
\newline
specifies if the connectivity of the wall cells must be freezed.
\newline
\newline
\hspace*{1cm} \texttt{.StandardShockFitting.MeshData.WithP0 = true}
\newline
\newline
specifies the name of the output file given by \texttt{COOLFluiD} according to the compiled \texttt{COOLFluiD} version.
Choose \texttt{true} for the \texttt{2013.9} version and \texttt{false} for the \texttt{2014.11} one or higher.
\newline
\newline
\hspace*{1cm}  \texttt{.StandardShockFitting.MeshData.NPROC = 4}
\newline
\newline
defines the number of processor used for the \texttt{COOLFluiD} execution. 
\newline
With \texttt{.NPROC = 1} it will be executed sequentially, with \texttt{.NPROC = 2} or more, it will be executed in parallel.
\newline
\newline
\hspace*{1cm} \texttt{.StandardShockFitting.MeshData.NBegin = 0}
\newline
\newline
specifies the number of the first step. If \texttt{.NBegin = 0} is chosen, the steps numbering will start from \texttt{0}. 
\newline
\newline
\hspace*{1cm} \texttt{.StandardShockFitting.MeshData.NSteps = 1000}
\newline
\newline
specifies for how many steps the simulation will be run.
\newline
\newline
\hspace*{1cm} \texttt{.StandardShockFitting.MeshData.IBAK = 100}
\newline
\newline
defines every how many steps the solution will be saved. The files are saved inside directories named \textit{step} and the number of the current step (e.g: the step number 101 will be saved in the folder named as \textit{step00101}).

\subsubsection*{PhysicsData}

\textbf{PhysicsInfo}
\newline
\newline
\hspace*{1cm} \texttt{.StandardShockFitting.PhysicsData.PhysicsInfo.NDIM = 2}
\newline
\hspace*{1cm} \texttt{.StandardShockFitting.PhysicsData.PhysicsInfo.NDOFMAX = 6}
\newline
\hspace*{1cm} \texttt{.StandardShockFitting.PhysicsData.PhysicsInfo.NSHMAX = 5}
\newline
\hspace*{1cm} \texttt{.StandardShockFitting.PhysicsData.PhysicsInfo.NPSHMAX = 1000}
\newline
\hspace*{1cm} \texttt{.StandardShockFitting.PhysicsData.PhysicsInfo.NESHMAX = 999}
\newline
\hspace*{1cm} \texttt{.StandardShockFitting.PhysicsData.PhysicsInfo.NADDHOLESMAX = 10}
\newline
\hspace*{1cm} \texttt{.StandardShockFitting.PhysicsData.PhysicsInfo.NSPMAX = 12}
\newline
\newline
specify the space dimension, the maximum number of degrees of freedom, the maximum number of shocks, the maximum number of shock points for each shock, the maximum number of shock edges for each shock\footnote{this values must always set equal to \texttt{NPSHMAX-1}}, the maximum number of holes, the maximum number of special points.
\newline
These options are mostly stable and should be not be changed at the first attempt.
\newline
\newline
\hspace*{1cm} \texttt{.StandardShockFitting.PhysicsData.PhysicsInfo.GAM = 1.40e0}
\newline
\newline
defines the value of the free-stream heat capacity ratio. This value is used only in the \texttt{PG} (\textit{Perfect Gas}) and \texttt{Cneq} (\textit{Chemical non equilibrium}) gas models.
\newline
\newline
\textbf{ChemicalInfo}
\newline
\newline
\hspace*{1cm} \texttt{.StandardShockFitting.PhysicsData.ChemicalInfo.Model = TCneq}
\newline
\newline
specifies the gas model. Up-to-date the \texttt{PG} (\textit{Perfet Gas}) and \texttt{Cneq} (\textit{Chemical non equilibrium} with argon mixture) and \texttt{TCneq} (\textit{Thermo-chemical non-equilibrium}) are implemented.
\newline
\newline
\hspace*{1cm} \texttt{.StandardShockFitting.PhysicsData.ChemicalInfo.Mixture = nitrogen2}
\newline
\hspace*{1cm} \texttt{.StandardShockFitting.PhysicsData.ChemicalInfo.InputFile = nitrogen2.dat}
\newline
\newline
define the name of the the gas mixture and the file containing the gas mixture informations. They must be specified only for the \texttt{TCneq} gas model.
\newline
The mixture file template is shown hereafter:
\newline
\newline
\hspace*{1.2cm}
\texttt{!NAME \hspace*{2cm} (name of the mixture)}
\newline
\hspace*{1.2cm}
\texttt{!NSP \hspace*{2.2cm} (number of the chemical species)}
\newline
\hspace*{1.2cm}
\texttt{!SPECIES \hspace*{1.4cm} (name of the species - IUPAC)}
\newline
\hspace*{1.2cm}
\texttt{!MM \hspace*{2.4cm} (molecular weight of the species $[$kg$/$mol$]$)}
\newline
\hspace*{1.2cm}
\texttt{!HF \hspace*{2.4cm} (formation enthalpy at 0 K of the species $[$J$/$kg$]$)}
\newline
\hspace*{1.2cm}
\texttt{!THEV \hspace*{2cm} (characteristic vibrational temperature $[$K$]$)}
\newline
\hspace*{1.2cm}
\texttt{!GAMS \hspace*{2cm} (specific heat ratio of each species)}
\newline
\hspace*{1.2cm}
\texttt{!TYPE \hspace*{2cm} (type of molecule:}
\newline
\texttt{\hspace*{6cm} A: atomic}
\newline
\texttt{\hspace*{6cm} B: di-atomic or aligned}
\newline
\texttt{\hspace*{6cm} T: tri-atomic non aligned)}
\newline
\newline
some examples can be found inside the folder \texttt{src$/$data$\_$template}
\newline
\newline
\hspace*{1cm} \texttt{.StandardShockFitting.PhysicsData.ChemicalInfo.Qref = /}
\newline
\newline
specifies the reference speed. It should be activated only for the \texttt{Cneq} model.
\newline
\newline
\textbf{ReferenceInfo}
\newline
\newline
\hspace*{1cm} \texttt{.StandardShockFitting.PhysicsData.ReferenceInfo.gamma = 1.4}
\newline
\hspace*{1cm}  \texttt{.StandardShockFitting.PhysicsData.ReferenceInfo.Rgas = 287.0e0}
\newline
\hspace*{1cm} \texttt{.StandardShockFitting.PhysicsData.ReferenceInfo.TempRef = 1833.0e0}
\newline
\hspace*{1cm} \texttt{.StandardShockFitting.PhysicsData.ReferenceInfo.PressRef = 57.65e0}
\newline
\hspace*{1cm} \texttt{.StandardShockFitting.PhysicsData.ReferenceInfo.VelocityRef = 5594.0e0}
\newline
\newline
are used by the \texttt{VariableTransformerSF} library and define the gas heat capacity ratio, the gas constant, the free-stream temperature, the free-stream pressure and the free-stream speed.
\newline
\newline
\hspace*{1cm} \texttt{.StandardShockFitting.PhysicsData.ReferenceInfo.isVelocityConcordantWithX = true}
\newline
\newline
specify if the x-component of the velocity is concordant with the x-axis. By default it is set to \texttt{false}.
\newline
\newline
\hspace*{1cm} \texttt{.StandardShockFitting.PhysicsData.ReferenceInfo.SpeciesDensities = \texttt{ \textbackslash{}}}
\newline
\hspace*{1.3cm} \texttt{0.00036354 0.00461646}
\newline
\hspace*{1cm} \texttt{.StandardShockFitting.PhysicsData.ReferenceInfo.Lref = 1.0e0}
\newline
\newline
define the species densities and a reference length. They must be specified only for the \texttt{TCneq} and \texttt{Cneq} models.

\subsubsection{MeshGeneratorSF}

\hspace*{1cm}\texttt{.StandardShockFitting.MeshGeneratorList = ReadTriangle ReSdwInfo \textbackslash{}}
\newline
\hspace*{9.5cm} \texttt{TriangleExe Tricall}
\newline
\newline
specifies the classes of the \texttt{MeshGeneratorSF} library called in the current model of the Shock-fitting Solver.
\newline
\newline
\hspace*{1cm} \texttt{.StandardShockFitting.ReadTriangle.InputFile = FILEPATH/na00.1}
\newline
\hspace*{1cm} \texttt{.StandardShockFitting.ReadTriangle.FileTypes = node poly ele neigh edge}
\newline
\newline
indicate the name  and the types of the mesh points reading files.
\newline
\newline
\hspace*{1cm} \texttt{.StandardShockFitting.ReSdwInfo.InputFile = FILEPATH/sh00.dat}
\newline
\newline
specify the name of the shock informations reading file.
\newline
If the \texttt{freezedWallcell} option is set to \texttt{true}, \texttt{ReadTriangleFreez} must be used in place of \texttt{ReadTriangle}.

\subsubsection*{RemeshingSF}

\hspace*{1cm} \texttt{.StandardShockFitting.RemeshingList = BndryNodePtr RdDpsEq FndPhPs
\textbackslash{}}
\newline
\hspace*{8.6cm} \texttt{ChangeBndryPtr CoPntDispl \textbackslash{}}
\newline
\hspace*{8.6cm}\texttt{ FixMshSps RdDpsEq}
\newline
\newline
specifies the classes of the \texttt{RemeshingSF} library called in the current model of the Shock-fitting Solver.
\newline
\newline
\hspace*{1cm} \texttt{.StandardShockFitting.CoNorm = CoNorm4TCneq}
\newline
\newline
defines the derived object of the \texttt{CoNorm} class that are asked to operate. It must be set according to the gas model: \texttt{Pg} or \texttt{Cneq} or \texttt{TCneq} should be added to the string \texttt{CoNorm4}.
\newline
If the \texttt{freezedWallcell} option is set to \texttt{true}, \texttt{BndryNodePtrFreez} must be used in place of \texttt{BndryNodePtr} and \texttt{BndryFacePtrFreez} must be added to the list.

\subsubsection*{WritingMeshSF}

\hspace*{1cm} \texttt{.StandardShockFitting.WritingMeshList = WriteTriangle \textbackslash{}}
\newline
\hspace*{9cm} \texttt{WriteBackTriangle \textbackslash{}}
\newline
\hspace*{9cm} \texttt{WriteSdwInfo }
\newline
\newline
specifies the classes of the \texttt{WritingMeshSF} library called in the current model of the Shock-fitting Solver.
\newline
If the \texttt{freezedWallcell} option is set to \texttt{true}, \texttt{WriteTriangleFreez} must be used in place of \texttt{WriteTriangle}.

\subsubsection*{ConverterSF}

\hspace*{1cm} \texttt{.StandardShockFitting.ConverterList = ShockFileConverter \textbackslash{}}
\newline
\hspace*{8.6cm} \texttt{CFmesh2StartingTriangle \textbackslash{}}
\newline
\hspace*{8.6cm} \texttt{Triangle2CFmesh CFmesh2Triangle}
\newline
\newline
specifies the classes of the \texttt{ConverterSF} library called in the current model of the Shock-fitting Solver. For each converter class the following lines must be specified (in the example below are related to \texttt{CFmesh2Triangle} class):
\newline
\newline
\hspace*{1cm} \texttt{.StandardShockFitting.CFmesh2Triangle.From = Prim}
\newline
\hspace*{1cm} \texttt{.StandardShockFitting.CFmesh2Triangle.To = Param}
\newline
\hspace*{1cm} \texttt{.StandardShockFitting.CFmesh2Triangle.GasModel = TCneq}
\newline
\hspace*{1cm} \texttt{.StandardShockFitting.CFmesh2Triangle.AdditionalInfo = Dimensional}
\newline
\newline
They define the strings that will create the name of the \texttt{VariableTrasformerSF} object asked to make the variables transformation. 
\newline
\newline
Up-to-date the \texttt{From} and the \texttt{To} options have \texttt{Prim} and \texttt{Param} as possible values. 
\newline
The \texttt{GasModel} can be \texttt{Pg} or \texttt{Cneq} or \texttt{TCneq}. 
\newline
The \texttt{AdditionalInfo} specifies the CFD variables format (\texttt{Dimensional} or \texttt{Adimensional}).
\newline
\newline
The \texttt{ShockCreatorFile}, \texttt{CFmesh2StartingTriangle}, \texttt{Triangle2CFmesh} objects have more informations in addition to the ones mentioned above.
\newline
The \texttt{ShockCreatorFile} creates the shock input file (\textit{sh00.dat}) from a \texttt{tecplot} poly-line tracing the shock profile.
\newline
\newline
\hspace*{1cm} \texttt{.StandardShockFitting.ShockFileConverter.InputFile = FILEPATH/shock.dat}
\newline
\newline
defines the name of the \texttt{tecplot} file containing the shock points poly-line.
\newline
\newline
\hspace*{1cm} \texttt{.StandardShockFitting.ShockFileConverter.nbDof = 6}
\newline
\hspace*{1cm} \texttt{.StandardShockFitting.ShockFileConverter.nbShocks = 1}
\newline
\hspace*{1cm} \texttt{.StandardShockFitting.ShockFileConverter.nbSpecPoints = 2}
\newline
\hspace*{1cm} \texttt{.StandardShockFitting.ShockFileConverter.TypeSpecPoints = OPY}
\newline
\newline
specify the options needed for the \textit{sh00.dat} file creation: the number of degrees of freedom, the number of shocks, the number of special points, the type of the special points.
\newline
Up-to-date only \texttt{OPY} can be chosen as special point.
\newline
\newline
The \texttt{CFmesh2StartingTriangle} is used to create the \textit{triangle} files from the starting captured solution.
\newline
\newline
\hspace*{1cm} \texttt{.StandardShockFitting.CFmesh2StartingTriangle.InputFile = FILEPATH/file.CFmesh}
\newline
\newline
specifies the name of the \texttt{COOLFluiD} file storing the captured solution.
\newline
\newline
The \texttt{Triangle2CFmesh} has an additional info that states if the shock boundary is \textit{single} or it is \textit{splitted} in a \textit{subsonic} and a \textit{supersonic} edges:
\newline
\newline
\hspace*{1cm} \texttt{.StandardShockFitting.Triangle2CFmesh.ShockBoundary = single}
\newline
\newline
the two options are therefore \texttt{single} or \texttt{splitted}.
\newline
If the \texttt{freezedWallcell} option is set to \texttt{true}, \texttt{Triangle2CFmeshFreez} and \texttt{CFmesh2TriangleFreez} must be used in place of \texttt{Triangle2CFmesh} and \texttt{CFmesh2Triangle}.
\newline
\newline
Converters from Tecplot format to \texttt{triangle} format are defined inside the code. When using the Residual Distribution Methods, the \texttt{Triangle2CFmesh} and \texttt{CFmesh2Triangle} converters can be replaced with \texttt{Triangle2Tecplot} and \texttt{Tecplot2Triangle}.
\newline

\underline{\emph{REMARK}}: when using the Finite Volume Method the called converters must be \texttt{Triangle2Tecplot}, \texttt{TecplotFVM2StartingTriangle} and \texttt{TecplotFVM2Triangle}.

\subsubsection*{CopyMakerSF}

\hspace*{1cm} \texttt{.StandardShockFitting.CopyMakerList = MeshBackup CopyRoeValues1 \textbackslash{}}
\newline
\hspace*{8.6cm} \texttt{CopyRoeValues2 MeshRestoring}
\newline
\newline
specifies the classes of the \texttt{CopyMakerSF} library called in the current model of the Shock-fitting Solver.

\subsubsection*{StateUpdaterSF}
\label{subsubsec:state updater}

\hspace*{1cm} \texttt{.StandardShockFitting.CopyMakerList = FixStateSps Interp $\backslash$}
\newline
 \hspace*{8.6cm} \texttt{ComputeResidual}
\newline
\newline
specifies the classes of the \texttt{StateUpdaterSF} library called in the current model of the Shock-fitting Solver.
\newline
\newline
\hspace*{1cm} \texttt{.StandardShockFitting.ComputeResidual.whichNorm = L1}
\newline
\newline
defines the norm of the discretization error used to compute the residual. Up-to-date the \texttt{L1} and \texttt{L2} norms are implemented.
\newline
\newline
\hspace*{1cm} \texttt{.StandardShockFitting.ComputeResidual.isItWeighted = true}
\newline
\newline
specifies if the norm is weighted on the first residual value.
\newline
\newline
\hspace*{1cm} \texttt{.StandardShockFitting.ComputeResidual.gasModel = Pg}
\newline
\newline
sets the gas model used to make the conversion to primitive variables.
\newline
\newline
\hspace*{1cm} \texttt{.StandardShockFitting.ComputeStateDps = ComputeStateDps4TCneq}
\newline
\hspace*{1cm} \texttt{.StandardShockFitting.MoveDps = MoveDps4TCneq}
\newline
\newline
define the \texttt{ComputeStateDps} and \texttt{MoveDps} objects asked to operate. They must be chosen according to the gas model: \texttt{Pg} or \texttt{Cneq} or \texttt{TCneq} must be added to the string \texttt{ComputeStateDps4} and \texttt{MoveDps4}.

\end{document}