\documentclass[11pt]{article}
%*******************************************************************
%--Packages
%*******************************************************************
\usepackage[final]{graphicx,epsfig}
%\usepackage{subfigure}
%\usepackage{subfigmat}
\usepackage{amssymb}                       % AMS symbols
\usepackage{amsmath}                       % AMSTeX Math typing
\usepackage{amsthm}                        % AMS theorem styles
\usepackage{amstext}                       % define \text for text in maths
\usepackage{listings}                      % print syntax-highlighted listings
%\lstloadlanguages{[ANSI]C++}              % activate ISO C++ language
\lstset{language=[ANSI]C++}
\usepackage{times}
\usepackage{graphicx}
\usepackage{subfigure}
\usepackage{color}

\newcommand{\noi}{\noindent}
\newcommand{\redish}[1]{\color{red}{#1}\color{black}}
\newcommand{\diff}{\mbox{\,d}}
\newcommand{\grad}{\mbox{$\nabla$}}
\newcommand{\divg}{\mbox{$\nabla\cdot$}}
\newcommand{\rot}{\mbox{$\nabla\times$}}
\newcommand{\pdiff}{\mbox{$\,\partial$}}
\newcommand{\f}{\frac}
\newcommand{\p}{\partial}
\newcommand{\be}{\begin{equation}}
\newcommand{\ee}{\end{equation}}
\newcommand{\bi}{\begin{itemize}}
\newcommand{\ei}{\end{itemize}}
\newcommand{\pch}{^{+}}
\newcommand{\dd}[2]{{\frac{\partial #1}{\partial #2}}}
\newcommand{\ddn}[2]{{\frac{d #1}{d #2}}}
\newcommand{\dddn}[2]{{\frac{d^2 #1}{d #2 ^2}}}
\newcommand{\omg}{\scriptscriptstyle\Omega}
\newcommand{\pdif}[2]{\frac{\partial #1}{\partial #2}}
\newcommand{\beq}{\begin{equation}}
\newcommand{\eeq}{\end{equation}}
\newcommand{\mb}[1]{\mathbf#1}
\newcommand{\smod}[1]{\mid#1\mid}

\hoffset-1.2cm
\voffset-2.0cm
\setlength{\textwidth}{15.5cm}
\setlength{\textheight}{22.5cm}

\renewcommand{\thepage}{\roman{page}}
\newtheorem{thm}{Theorem}[section]
\newtheorem{Assumption}{Assumption}
\newtheorem{Proposition}{Proposition}
\newtheorem{Remark}{Remark}


\def\div{\mathop{\rm div}}
\def\divB{\divergence {\bf B}}
\def\divergence{\mathop{\nabla\cdot}}
\def\divergencewr{\mathop{{\rm div}}}
\def\divb{\divergencewr({\bf B})}
\def\divbn{\divergence{\bf B}}

% Predefined colors are: white, black, red, green, blue, cyan, magenta, yellow
% Define own dark blue:
\definecolor{darkblue}{rgb}{0,0,0.7}
\setlength{\parindent}{0pt}
\setlength{\parskip}{5pt plus 2pt minus 1 pt}
\topmargin  -5mm
\evensidemargin 8mm
\oddsidemargin  2mm
\textwidth  158mm
\textheight 230mm
%\renewcommand{\baselinestretch}{1.0}
\frenchspacing
\sloppy


%%%%%%%%%%%%%%%%%%%%%%%%%%%%%%%%%%%%%%%%%%%%%%%%%%%%%%%%%%%%%%%%%%%%%%%

\begin{document}
\pagestyle{empty}

\begin{center}
  {\fontsize{14}{20}\bf
    User/Developer's Manual for ShockFitting (SF): \\
    {\small A modular shock-fitting code for unstructured grids.} \\[10pt]}
\end{center}

\begin{center}
  Valentina De Amicis, \underline{valentina.deamicis91@gmail.com}\\
  Dr. Andrea Lani, \underline{lani@vki.ac.be}\\
  Prof. Renato Paciorri, \underline{paciorri@dma.ing.uniroma1.it}\\
  Prof. Aldo Bonfiglioli, \underline{aldo.bonfiglioli@unibas.it}
\end{center}

%%%%%%%%%%%%%%%%%%%%%%%%%%%%%%%%%%%%%%%%%%%%%%%%%%%%%%%%%%%%%%%%%%%%%%%%%%%%%%

\section*{Introduction}

SF is a modular shock fitting solver that can couple to arbitrary unstructured CFD codes.

\section{Configuration file}

The input file for the code is in a readable text format, where keywords are associated 
to values according to a dynamic configuration language, which is described hereafter.

\subsection{Format description}

\subsubsection{Full format (discouraged)}\label{ssec:full_format}

The format consists of lines in the form:
\vspace{-0.2cm}
\begin{verbatim}
.OptionA = Value1               # use Value1 as value for OptionA
.Value1.OptionB = Value2        # use Value2 as value for OptionB
.Value1.Value2.OptionC = Value3 # use Value3 as value for OptionC
.Value1.Value2.OptionD = Value4 # use Value4 as value for OptionD
\end{verbatim} 
where, in each line, the whole LHS is the keyword and the RHS is the value. The
latter can be:

\begin{itemize}
\item
  an alpha-numerical string
\item
  an integer
\item 
  a boolean (\texttt{true} or \texttt{false})
\item
  a floating point number
\item
  an arbitrarily complex analytical function
\item
  an array of all the previous.
\end{itemize}
The keyword is composed of literal strings separated by ".",
corresponding to different \textit{entities} (configurable
objects or parameters) defined inside the actual code. The
configuration is hierarchical and recursive from top to lowest
level. The order in which the options are declared in the file are
irrelevant. If needed, the value can be broken into different lines by using the continuation character (back slash) 
at the end of each line (note that the number of spaces at the end or
before the line is irrelevant):
\vspace{-0.2cm}
\begin{verbatim}
.Example.arrays = 4 4 10 \
                 10 4 \
                 4 3
\end{verbatim}
Comments start with "\#": they can occupy full lines or be
placed at the end of the line. Full blocks of comments can be enclosed 
between a starting and ending \textbf{single} "\$", without needing a "\#" in each
line, as in the example below:
\vspace{-0.2cm}
\begin{verbatim}
$
block of comments can be conveniently 
written in this way without needing
a "#" character in each line 
$
\end{verbatim} 

\subsubsection{Compact format (standard)}\label{ssec:compact_format}

The full format example in Sec. \ref{ssec:compact_format} can be
significantly simplified:

\vspace{-0.2cm}
\begin{verbatim}
.OptionA = Value1 
.Value1 {   # start block of nested options for Value1 (note the ".")
 .OptionB = Value2 
 .Value2 {  # start block of nested options for Value2 (note the ".")
 .OptionC = Value3 ; OptionD = Value4   # semicolon separates options
 }    # this brace ends the block for OptionB (Value2)
}     # this brace ends the block for OptionA (Value1)
\end{verbatim} 
where, in each line, the LHS of the "=" is the keyword and the RHS is
the value. Each block of options starts with "{" right after the last
  value and ends with "}". The latter must be the only character on
the line, a part from comments, if any. Note that with the compact 
format, different options can be put \textbf{within the same line}, 
separated by ";".  While reading the configuration file 
(see implementation details in SConfig/src/ConfigFileReader.cxx), 
the compact format is internally converted into the full format.

\subsubsection{Errors handling}

The file parser is able to detect errors of different kind. It will
raise an exception and abort the code if one the following situations
occur:

\begin{itemize}
\item
  a keyword (explicitly indicated within "$<>$") is duplicated;
\item
  a keyword (explicitly indicated within "$<>$") doesn't exist;
\item 
  a value is not satisfying some predefined validating conditions.
\end{itemize}
If the keyword includes (between two successive ".") the name of a
polymorphic object (base class or derived class), the corresponding
option, if defined in the parent class (and therefore inherited by the
derived ones) can be addressed by both the derived class name or the 
base class name. {\bf All fields previously defined correspond to base
  classes}, where most of the options are defined. 

\subsubsection{Warnings}\label{sssec:warnings}

Let's consider the following example:
\vspace{-0.2cm}
\begin{verbatim}
.BaseObject = DerivedObject  # DerivedObject is a derived class of BaseObject
.BaseObject.pnr = 70         # pnr is an option defined in BaseObject
.DerivedObject.pnr = 40      # but it is also inherited (and "visible") by DerivedObject 
\end{verbatim}
If you specify two options for the same parameter (\texttt{pnr} in our
case), one with the parent \texttt{BaseObject} and another with the actual
(derived) name \texttt{DerivedObject}, the parser will not detect an error but 
will notify a self-explanatory warning:
\vspace{-0.2cm}
\small
\begin{verbatim}
WARNING: Parameter < pnr > is DUPLICATED: is this what you really expect?
\end{verbatim}
In our example, there is actually an error to fix, since it is not
clear which value the user wants to assign to \texttt{pnr}. If not
properly fixed, the error could lead to undefined behaviour. \\
Let's make another example where the warning would actually be harmless:
\vspace{-0.2cm}
\begin{verbatim}
.DerivedObject.Inner.Funs = 181.38 \
                         1936.18*cos(25./180*pi) \
                         0. \
                         1936.18*sin(25./180*pi) \
                         259.96

.DerivedObject.BcField.Funs = 181.38 \
                         1936.18*cos(25./180*pi) \
                         0. \
                         1936.18*sin(25./180*pi) \
                         259.96
\end{verbatim}
In this case, two distinct fields \texttt{Inner} and
\texttt{BcField} (not deriving one from each other!)
have been defined, each one specifying an option named \texttt{Funs},
corresponding to an array of user-defined functions, one per flow 
variable. Even though \texttt{Funs} is used twice and all 
corresponding values are the same, the two keywords refer to 
different entities and is perfectly normal to get the above
mentioned warning.

\subsubsection{Log files}\label{sssec:logfiles}

After parsing the configuration file, two output files are written 
(only by the process with rank 0 in a paralllel simulation):
\begin{itemize}
\item
  config.log, where all detected options key and values are listed; 
\item
  options.dat, which dumps all descriptions associated to the
  option keys corresponding to the selected simulation fields.
\end{itemize}

\subsubsection{Interactive parameters}

All parameters that have been declared interactive (see
following section) will be updated after a number of iterations
indicated by the user (which is also by itself interactively modifiable):
\vspace{-0.2cm}
\begin{verbatim}
.config_rate = 1000 # every 1000 iterations the code updates the configuration
\end{verbatim}

\newpage
\section{SConfig: a configuration library (by Andrea Lani)}\label{ssec:sconfig}

The parsing of the configuration file and its interpretation is
performed by a newly implemented library, that we have denominated
\textit{SConfig}. The latter combines concepts inspired by 
\cite{Beveridge} (self-registering objects) and \cite{Yagol} 
(object configuration) and more sophisticated techniques 
described in \cite{Quintino08, Lani08} (self-registering
and self-configuring objects) into a new implementation. \\
The SConfig library is completely independent from ADPDIS3D 
and can be linked statically or dynamically to whatever other code to
take profit of the very general functionalities it provides.\\
Herein, while the C++ implementation details are left out of the present document,
the usage of the library from a developer point of view will now be 
addressed.

\subsection{Self-configuring objects}\label{ssec:self-configuration}

In order to take profit of the self-configuration capability, 
the developer must define \textbf{objects deriving from \texttt{ConfigObject}} 
or from other objects already deriving from it. The interface of 
\texttt{ConfigObject} is shown in listing \ref{clist:ConfigObject}.     
\begin{lstlisting}[basicstyle={\small\sffamily},float=!htb,caption={ConfigObject interface.},
  label=clist:ConfigObject]
  class ConfigObject : public NamedObject {
    public: 
    
    // constructor
    ConfigObject(const string& name);
    
    // virtual destructor
    virtual ~ConfigObject();
    
    // set the parameter corresponding to the following inputs
    template <typename TYPE>
    void addOption(const string& optionName, 
                   TYPE* optionPtr, 
                   const string& optionDescription, 
                   bool isDynamic = false,
                   OptionValidation<TYPE>* condition = NULL);
    
    // configure all parameters from file
    virtual void configure(OptionMap& cmap, string* prefix = NULL);
    
    // configure dependent objects
    virtual void configureDeps(OptionMap& cmap, ConfigObject *const other);
    
  protected:
     // get the name of the parent
     virtual string getParentName() const = 0;

   ... // private data follows
  };
\end{lstlisting}
\noi
Here follows some guidelines for activating the self-configuration
capability.
\begin{enumerate}
\item
  An object deriving from \texttt{ConfigObject} must implement the pure virtual
  function \texttt{getParentName()} specifying the name of the parent 
  class. This allows users to specify one of both the parent class name or the
  derived class name for configurable options defined in the parent (see
  Sec. \ref{sssec:warnings} for an example), in order not to have to
  introduce too many changes in new configuration files when adapting
  them for new cases.
\item
  A \texttt{ConfigObject} child needing to declare some options
  must \textbf{call explicitly the \texttt{addOption()} function for each
    single option} inside its own constructor, specifying the
  appropriate arguments to it, namely:
  \begin{itemize}
  \item the option name;
  \item a pointer to the local (in the class) variable to be linked to
    the option;
  \item the option description (indicating the purpose of the option);   
  \item a boolean (\texttt{true} or \texttt{false}) to specify if the
    option is eligible to be modified interactively during the
    simulation (default value is \texttt{false});
  \item a pointer to an object deriving from \texttt{OptionValidation}
    responsible for checking if the option value is valid.    
  \end{itemize}
  The last two function arguments are optional and can be omitted. 
  However, the fifth argument cannot be specified if the fourth 
  isn't present (this is due to C++ treatment of default arguments).
\item 
  A \texttt{ConfigObject} child can in general omit to implement 
  the function \texttt{configure()}, unless something special is
  needed during the configuration phase, like configuring nested 
  objects (to which the object in question holds pointer) via 
  \texttt{configureNested()}. \textbf{Inside the \texttt{configure()} implementation, a
    call to \texttt{configure()} on the closest parent 
    class must be executed first}.
\item
  The \texttt{configureNested()} function defined in 
  \texttt{ConfigObject} should not be overridden by derived 
  classes (even though such a possibility is left open in case a 
  SConfig user really need to do something extraordinary atypical).
\end{enumerate}

\subsection{Self-registering objects}\label{ssec:self-registration}

The self-registration technique pioneered in \cite{Beveridge} allows 
polymorphic objects (i.e. deriving from some existing base classes) to be 
dynamically integrated into an existing code, without having to modify
a single line of the latter. The technique effectively replaces the
use of conditional constructs \texttt{if-else-if} or \texttt{switch}
in a more elegant and flexible way. SConfig provides a generic 
self-registration capability which can be enabled as explained hereafter.
\begin{enumerate}
\item
  The self-registration affects all objects belonging to the same 
  hierarchy, or, in other words, with the same polymorphic type, which
  is the same as the base class type. In a generic base class
  \texttt{Base} of a hierarchy of potentially self-registering objects, 
  one must define the type \texttt{PROVIDER}, the type \texttt{ARGS} 
  (constructor arguments) and the class name, as shown in listing \ref{clist:SelfRegistration}.  
\begin{lstlisting}[basicstyle={\small\sffamily},float=!htb,caption={How
    to enable Self-registration.}, label=clist:SelfRegistration]
  class Base {
    public:
      typedef Provider<Base> PROVIDER;  // definition of the type PROVIDER 
      typedef ArgumentsType ARGS;       // definition of the type ARGS      

      Base(ArgumentsType args);         // constructor taking ArgumentsType    
      static string getClassName(){return "Base";}   // define class name

      // ... whole class definition follows
  };
\end{lstlisting}
\noi
In our implementation, \texttt{Base} constructor must accept one single
argument which can be of generic type. In case more than one argument
is needed, \texttt{ArgumentsType} can be a \texttt{std::pair} or tuple
object grouping multiple arguments (or even none) of whatever type, 
for instance:
\begin{lstlisting}[basicstyle={\small\sffamily},float=!htb,caption={Examples
    of argument tuples.}, label=clist:ArgTuples]
  template <typename T1, typename T2, typename T3>
  struct Args0 {};

  template <typename T1, typename T2, typename T3>
  struct Args3 {T1 arg1; T2 arg2; T3 arg3;};
\end{lstlisting}
\noi
\item
  All objects to be made self-registering must derive from a
  base class satisfying the previous requirement and must define a
  global variable of type  \texttt{ObjectProvider} parameterized with 
  the derived class static and polymorphic types in a source file 
  (not on an header!). An illustrative example is shown in
  \ref{clist:ObjectProvider}.
  
\begin{lstlisting}[basicstyle={\small\sffamily},float=!htb,caption={ObjectProvider
    defined in DerivedClass.cxx.}, label=clist:ObjectProvider]
  ObjectProvider<MyDerived, Base> myDerivedProvider("MyDerived1");
\end{lstlisting}
\noi
The provider associates a name to a polymorphic object (in this case
"MyDerived1" to \texttt{MyDerived}) allowing the code to construct
such an object on demand, just by knowing its name.
\end{enumerate}

\subsection{A clarifying example} \label{ssec:example}

After having explained how to make an object self-configurable or 
self-registering, we are now ready to analyze a case where the two
properties are combined together, in order to get maximum profit of the 
two techniques. Let's consider the base class \texttt{BaseObject} 
(actually defined inside ADPDIS3D), whose interface is presented 
in listing \ref{clist:BaseObject}.

\begin{lstlisting}[basicstyle={\small\sffamily},float=!htb,caption={BaseObject
    class definition.}, label=clist:BaseObject]
#include <ConfigObject.hh>
#include <Provider.hh>

class BaseObject : public ConfigObject {
public:
  typedef Provider<BaseObject> PROVIDER;
  typedef const string& ARGS; // constructor takes a string (name) as input
  
  BaseObject(const string& name); // constructor
  virtual ~BaseObject() {}        // destructor
  
  // ... member virtual some functions
  
  // get the class name
  static string getClassName() {return "BaseObject";}

protected:
  std::string getParentName() const {return getClassName();}
};  
\end{lstlisting}
\noi
On the one hand, \texttt{BaseObject} derives from \texttt{ConfigObject}, which makes it
self-configurable, on the other hand, it defines the types
\texttt{PROVIDER}, \texttt{ARGS} and the class name, which make itself
and its children eligible for self-registration. \\
Let's now consider a prototype derived class \texttt{DerivedObject}, 
implementing a specific kind of problem for multiple overlapping 
meshes, as defined in listing \ref{clist:DerivedObject}. 
\begin{lstlisting}[basicstyle={\small\sffamily},float=!htb,caption={DerivedObject
    class definition.}, label=clist:DerivedObject]
class DerivedObject : public BaseObject {
public:
  DerivedObject(const string& name); // constructor
  ~DerivedObject();                  // destructor
  
  // ... some member functions 
 
  // configure all parameters from file
  void configure(OptionMap& cmap, std::string* prefix);
  
private:
  vector<double> m_inletState;  // state vector defining the inlet
  int m_nspecies;               // number of species
};
\end{lstlisting}
\noi
In the file DerivedObject.cxx, we can instantiate the corresponding provider by defining:
\begin{lstlisting}[basicstyle={\small\sffamily},float=!htb,caption={ObjectProvider
    for DerivedObject defined in DerivedObject.cxx.}, label=clist:DerivedObjectProvider]
  ObjectProvider<DerivedObject, BaseObject> oversetProvider("DerivedObject");
\end{lstlisting}
\noi
This provider defines the name "DerivedObject" which qualifies a specific
kind of \texttt{BaseObject} and which can be used as a keyword or as a
value in the configuration file to configure the corresponding 
object (see examples in Sec. \ref{sssec:warnings}).\\
We can now qualify the private data of \texttt{DerivedObject} to be
configurable by following the guidelines described in 
Sec. \ref{ssec:self-configuration} and adding the corresponding
options inside the constructor.
\begin{lstlisting}[basicstyle={\small\sffamily},float=!htb,caption={DerivedObject
    constructor.}, label=clist:DerivedObjectConstrutor]
DerivedObject::DerivedObject(const string& name) : BaseObject(name)
{  
  m_inletState = vector<double>();  // default value for inlet field
  addOption("inlet",&m_inletState,"State for inlet field", true,
            new GenericValidation< vector<double> >("if(w>0.,1,0)"));
            
  m_nspecies = 0;              // default value for number of species
  addOption("species",&m_species,"Number of species", false, 
      new CrossCheckValidation<int>
      ("if((w>0&w1=4&w2>0)|(w=0&w1!=4&w2>=0),1,0)","equation,implicit"));
}
\end{lstlisting}
\noi
Herein, two configurable options (key names and corresponding values) are defined:
"inlet" and "species", of type \texttt{vector<double>} and \texttt{int}
respectively. The value of "species" cannot be changed interactively 
during the simulation (fourth argument of \texttt{addOption()}
is set to \texttt{false}), while the inlet state can be modified
interactively, allowing time dependent inlet conditions. \\
In both cases, we focus our attention on the last parameters which 
define the option validation conditions. \\
\texttt{GenericValidation} accepts a string specifying one
arbitrarily complex analytical function \texttt{F} in the form
\begin{verbatim}
if(F(w),1,0)
\end{verbatim}
where "w" indicates the option value (which will be set in the input
file) and "1" stands for "true" and 0 for "false". In the first simple
case (here included only for illustrative purposes), we ask that each individual 
component of \texttt{m\_inletState} is positive:
\begin{verbatim}
if(w>0.,1,0)
\end{verbatim}
We remark that the expression can be much more complex than this,
involving analytical mathematical functions (abs, sqrt, tan, sin,
etc.) and constants (pi, e, etc.). \\
The second example better shows the power of this approach.
\texttt{CrossCheckValidation} takes two strings: the first string defines
the analytical validation function
\begin{verbatim}
if((w>0&w1=4&w2>0)|(w=0&w1!=4&w2>=0),1,0)
\end{verbatim}
where "\&" and "|" stands for "and" and "or"; the second string
specify a comma separated list of other existing configurable 
options ("equation,implicit" in our example) which have been 
defined in the class itself or in whatever other configurable object
existing in the code. Those options name are mapped to variable names
of the type "wN" where N is a number corresponding to the position 
in the specified name list. As a result,  "w1" corresponds to "equation", "w2"
corresponds to "implicit" in the validation expression 
\begin{verbatim}
if((w>0&w1=4&w2>0)|(w=0&w1!=4&w2>=0),1,0)
\end{verbatim}
allowing developers to cross check validity of one user-input value
(the value corresponding to "species"), in comparison with other related 
inputs (the values corresponding to "equation" and "implicit").\\
The input and parsing of analytical functions is made possible by the
open source FunctionParser library (version 4.0.5) \cite{FParser} 
available under LPGL license, whose full sources and licence files are
included in the current ADPDIS3D distribution and automatically
pre-compiled into a separate statically linked library.

\subsection{User-defined anaytical fields}

Let's now elaborate on the example in \ref{ssec:example} to show how anaytical
functions turn out to be useful also in different contexts inside 
ADPDIS3D, for instance to input generic space/time varying boundary 
conditions or initial fields. We substitute the constant initial 
values defined inside \texttt{DerivedObject} with an array of generic 
initial fields, one per conservative (or other useful) variable (see
listing \ref{clist:DerivedObject2}).
\begin{lstlisting}[basicstyle={\small\sffamily},float=!htb,caption={DerivedObject
    class definition 2.}, label=clist:DerivedObject2]
class DerivedObject : public BaseObject {
  // ... everything as before
  private:
    int m_nspecies;                     // number of species
    vector<string> m_iniNames;          // names of initialization fields
    vector<FunctionField*> m_iniFields; // initialization fields functions
    // other useful private data (check the code ...)
};
\end{lstlisting}
\noi
Herein, the variable $m\_iniNames$ defines the names for fields for
which a specific initialization wants to be imposed: the first name
(user-defined) {\bf must} correspond to the internal portion of the 
domain ("Inner", "Domain", or whatever else);
the other names must be the boundary IDs corresponding to the
boundary conditions specified in the mesh file (e.g. "100" for
Dirichlet, "8" for periodic boundary etc., as defined in the file
bc.F) if the user wishes to specify some specific initial values. Each name
is associated to a different \texttt{FunctionField}, 
a wrapper class for the  FunctionParser \cite{FParser} defined in 
ADPDIS3D enabling the end-user to input multiple anaytical functions, 
one per flow variable. The implementation details are left to the 
interested reader. We look at the effect on a prototype
configuration file:
\begin{verbatim}
.DerivedObject.IniFields = Inner 

.Inner {                     # initialization field names
                             # specific settings for the field called "Inner"
.InVars = fact
.InDefs = if(R2(x,y)<=0.5,0.,if(R2(x,y)>0.9,1.,(R2(x,y)-0.5)/0.4))
.Funs = 0.01*(fact+7.*(1.-fact)) \      # rho
                 	42.7021*FC 0. 0. \    # rhoU rhoV rhoW
	                92956.5597205 0. 0. 0.  # rhoE Bx By Bz
}
\end{verbatim}
We have declared some parameters for the initialization field, namely
\texttt{InVars}, \texttt{InDefs} and \texttt{Funs}:
\begin{itemize}
\item
  \texttt{InVars} allows the user to enter the names of some new 
  variables that will be used in the expressions specified by \texttt{Funs}. By
  default (if \texttt{InVars} is omitted), \texttt{Funs} can be
  formulated only in terms of spatial coordinates (x,y,z) and t (time).
\item
  \texttt{InDefs} are the functions defining the variables declared by
  \texttt{InVars}. Those functions can only be using x,y,z,t as 
  indipendent variables. The number of \texttt{InVars} defined, separated
  by a space, must match the number of \texttt{InDefs}.
\item
  \texttt{Funs} defines functions of x,y,z,t and the extra variables
  named in \texttt{InVars}. 
\end{itemize}
If we look at the given example, the user has defined an extra variable called \texttt{fact} as
\begin{verbatim}
if(R2(x,y)<=0.5,0.,if(R2(x,y)>0.9,1.,(R2(x,y)-0.5)/0.4))
\end{verbatim}
where \texttt{R2(x,y)} computes the distance in 2D (\texttt{R2(x,y,z)}
should be used in a 3D case). The final expressions of the initial
field, one per conservative variable (=8 in total), are functions of x,y,z,t and
\texttt{fact}, as shown above, and they are clearly much simpler than what they would
be without the definition of the intermediate \texttt{fact}
variable. \\
In our example we have restricted ourselves to an initialization
field, but a similar treatment is reserved to boundary condition
fields as well. \\
As far as the FunctionParser is concerned, syntax errors in the
analytical functions, if present, will be detected and notified to the
user. In particular, \textbf{no spaces are allowed inside an
  analytical function.} \\
An extensive list of  \textbf{predefined supported functions} is available
directly on the FunctionParser official webpage: 
\textit{http://warp.povusers.org/FunctionParser/fparser.html\#parsertypes}.
More functions and mathematical constants can be defined inside the
header file \texttt{src/CPP/DerivedParser.hh}, provided within this 
distribution: interested developers can check the file itself 
for examples on how \texttt{R2}, \texttt{R3} and some constants
(\texttt{pi}, \texttt{e}, \texttt{Rair}) were added.

\begin{thebibliography}{00}
  
\bibitem{Beveridge}
  J. Beveridge. Self-Registering Objects in C++. {\em Dr. Dobb's
    Journal}, 8/1998.
  
\bibitem{Yagol}
  J. Pace. Another Getopt Library, http://yagol.sourceforge.net, 2003.
  
\bibitem{Lani08}
  A.~Lani. \newblock {\em An Object Oriented and high performance platform for
    aerothermodynamics simulation}.
  \newblock PhD thesis, Von Karman Institute for Fluid Dynamics, 2008.
  
\bibitem{Quintino08}
  T. L.~Quintino. \newblock {\em A Component Environment for High-Performance
    Scientific Computing. Design and Implementation}.
  \newblock PhD thesis,  Von Karman Institute for Fluid Dynamics, 2008.
  
\bibitem{FParser} 
  J. Nieminen, J. Yliluoma. Function Parser for C++, 
  http://warp.povusers.org/FunctionParser/, 2009.
  
\end{thebibliography}

\end{document}
